\section{Discussion}
\label{sec:discussion}

Leading key-value stores today rely on LSM-DS methodology for serving requests mostly from RAM.
With this approach, the implementation of in-memory building blocks is critical for performance, as we have demonstrated in Section~\ref{sec:eval}.
The primary challenge such systems face is scaling up with the available hardware resources – most notably, the number of CPU cores.
In this context, the concurrency control that protects shared data structures can be a major performance roadblock.
Our work overcomes this roadblock and presents \clsm, an efficient concurrent LSM-DS implementation.
Scalability is achieved  by eliminating blocking  in scenarios that do not involve physical access to disk.
%Scalability is achieved  by eliminating blocking  in all scenarios that do not involve access to disk.

In addition to atomic reads and writes, \clsm\ supports consistent snapshot scans, range queries, and atomic read-modify-write operations.
Our algorithm is generic, and can be applied to a range of  implementations. Such decoupling allows our solution to be
combined with other optimization applied to the disk components and merge utility.


Our evaluation versus state-of-the-art LSM implementations shows performance improvements and superior scalability,
even when the competitors utilize smaller partitions. The latter, along with other disadvantages of partitioning discussed
in Section~\ref{sec:background}, suggests that our approach can potentially serve as an alternative for vertical scalability.


%The implementation of in-memory building blocks is critical for the
%performance of LSM-DS (as demonstrated in Section~\ref{sec:eval}). The primary challenge is scaling up with the available
%hardware resources -- in particular, the number of CPU cores.
%Similar to all systems that capitalize on internal parallelism, the concurrency control
%that protect the shared data structures is either a major
%performance roadblock or requires nontrivial mechanisms to guarantee the
%overall correctness of the system.
%In this work, we present an efficient concurrent LSM-DS implementation. It features atomic reads and
%writes, snapshots, iterators and an additional atomic read-modify-writes. Our algorithm is
%non-blocking in scenarios that do not involve access to disk.
%The algorithm for supporting the basic key-value store API is generic and can be
%applied to any implementation that does not exploit concurrency.
%Furthermore, this decoupling allows our solution to be combined with
%other optimization that are applied to the disk components and merge utility.
%
%Our evaluation versus state-of-the-art LSM implementations show performance
%improvements and superior scalability, even when the competitors utilize
%smaller partitions. This suggest that our concurrent algorithm can serve as an
%alternative for gaining vertical scalability.

