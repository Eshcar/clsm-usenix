%% $Id: genericmacros.tex,v 1.1 2008/11/06 22:41:17 eyahav Exp $
% Generic macros (nothing specific to this paper)

\usepackage{ifthen}
\usepackage{multirow}
%

%%%%%%%%%%%%%%%%%%%%%%%%%%%%%%%%%%%%%%%%%%%%%%%%%%%%%
%% Terminology                                     %%
%%%%%%%%%%%%%%%%%%%%%%%%%%%%%%%%%%%%%%%%%%%%%%%%%%%%%
\newcommand{\demonsmarker}{\textbf{\color{blue} DEMONS START HERE IF NOT EARLIER}}

%\newcommand{\scode}[1]{{ \texttt{#1}}}
%\newcommand{\sname}[1]{{\small \textsc{#1}}}

\newcommand{\TODO}[1]{{{\color{blue}{\bf #1}} }}
\newcommand{\TODOFN}[1]{ \footnote{{\bf #1}} }
\newcommand{\DONE}[1]{}
\newcommand{\COMMENT}[1]{}
\newcommand{\ie}{i.e.}
\newcommand{\eg}{e.g.}
%%% references definition
\newcommand{\lnref}[1]{Line~\ref{Ln:#1}}
\newcommand{\lemref}[1]{Lemma~\ref{Lm:#1}}
\newcommand{\figref}[1]{Figure~\ref{Fi:#1}}
\renewcommand{\algref}[1]{Algorithm~\ref{Al:#1}}
\newcommand{\defref}[1]{Definition~\ref{De:#1}}
\newcommand{\theref}[1]{Theorem~\ref{Th:#1}}
\newcommand{\tabref}[1]{Table~\ref{Ta:#1}}
\newcommand{\secref}[1]{Section~\ref{Se:#1}}
\newcommand{\secrefs}[2]{Sections~\ref{Se:#1}-\ref{Se:#2}}
\newcommand{\ssecref}[1]{Section~\ref{Se:#1}}
\newcommand{\appref}[1]{Appendix~\ref{Se:#1}}
\newcommand{\exref}[1]{Example~\ref{Ex:#1}}
\newcommand{\chapref}[1]{Chapter~\ref{Ch:#1}}
\newcommand{\equref}[1]{Equation~\ref{Eq:#1}}
\newcommand{\corref}[1]{Corollary~\ref{Co:#1}}

\newcommand{\lemlabel}[1]{\label{Lm:#1}}
\newcommand{\figlabel}[1]{\label{Fi:#1}}
\newcommand{\deflabel}[1]{\label{De:#1}}
\newcommand{\thelabel}[1]{\label{Th:#1}}
\newcommand{\tablabel}[1]{\label{Ta:#1}}
\newcommand{\seclabel}[1]{\label{Se:#1}}
\newcommand{\sseclabel}[1]{\label{Se:#1}}
\newcommand{\applabel}[1]{\label{Se:#1}}
\newcommand{\exlabel}[1]{\label{Ex:#1}}

%%\newtheorem{Examp}{Example}
\newtheorem{definition}{Definition}
%%% sectioning


\newcounter{programlinenumber}
\newenvironment{pgm}
     {\tt%
       \begin{tabbing}%
       123\=123\=123\=123\=123\=123\=123\=123\=123\=123\=123\=123\=\kill%
       \setcounter{programlinenumber}{0}%
     }
     {\end{tabbing}}
\newcommand{\nl}{\addtocounter{programlinenumber}{1}[\arabic{programlinenumber}] \> }
\newcommand{\nonl}{}%
\newenvironment{narrowpgm}
     {\tt%
       \begin{tabbing}%
       12\=12\=12\=12\=12\=12\=12\=12\=12\=12\=12\=12\=\kill%
       \setcounter{programlinenumber}{0}%
     }
     {\end{tabbing}}

\newenvironment{basenarrowpgm}
     {\tt%
       \begin{tabbing}%
       \=12\=12\=12\=12\=12\=12\=12\=12\=12\=12\=12\=12\=\kill%
       \setcounter{programlinenumber}{0}%
     }
     {\end{tabbing}}


\newenvironment{xbasenarrowpgm}
     {\tt%
       \begin{tabbing}%
       1\=1\=1\=1\=1\=1\=1\=1\=1\=1\=1\=1\=1\=\kill%
       \setcounter{programlinenumber}{0}%
     }
     {\end{tabbing}}

%% if statement
\newcommand{\ifthen}[3]{
 \left\{
      \begin{array}{ll}
        #2 & #1 \\
        #3 & {\rm otherwise}
      \end{array}
      \right.
 }

%% TO Allow writing the TR in the same source
\newboolean{TR}
\setboolean{TR}{false}
\ifthenelse{\boolean{TR}}{
\newcommand{\TrSelect}[2]{#1}
\newcommand{\TrOnly}[1]{#1}
\newcommand{\SubOnly}[1]{}
\newcommand{\TrOnlyInFootnote}[1]{#1}
\newcommand{\TrOnlyInTable}[1]{#1}}
{
\newcommand{\TrSelect}[2]{#2}
\newcommand{\TrOnly}[1]{}
\newcommand{\SubOnly}[1]{#1}
\newcommand{\TrOnlyInFootnote}[1]{}
\newcommand{\TrOnlyInTable}[1]{}}



%% General macros
\newcommand{\Set}[1]{\{ \; {#1} \; \}}
\newcommand{\vbar}{\; | \;}
\newcommand{\notIn}{\not\in}
\newcommand{\sizeof}[1]{|{#1}|}
%% \newcommand{\isDefined}{=_{def}}
\newcommand{\isDefined}{\triangleq}

\newcommand{\lt}{$<$}
\newcommand{\gt}{$>$}

%% tminus: "thin minus", with less space around it;
%% telem: "thin element"
\newcommand{\tminus}{\!\! - \!\!}
\newcommand{\telem}{\! \in \!}

%% stexttt: "small" texttt
\newcommand{\stexttt}[1]{{\small \texttt{#1}}}

\newcommand{\mathify}[1]{\mathord{\mbox{#1}}}
\newcommand{\italMathId}[1]{\mathify{\textit{\textrm{#1}}}}
\newcommand{\romanMathId}[1]{\mathify{\textup{\textrm{#1}}}}
\newcommand{\ttMathId}[1]{\mathify{\textup{\texttt{#1}}}}

\newcommand{\ttsub}[2]{${\ttMathId{#1}}_{#2}$}
\newcommand{\ttsup}[2]{${\ttMathId{#1}}^{#2}$}
\newcommand{\mttsub}[2]{{\ttMathId{#1}}_{#2}}
\newcommand{\twoarray[2]}{\begin{array}{l}  #1 \\ #2 \end{array}}
\newcommand{\subsubsubsection}[1]{\emph{#1}:}
